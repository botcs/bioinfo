\chapter*{Literature}
\addcontentsline{toc}{chapter}{Literature}

A humble listing of the selected bibliography and related works, including their \emph{Abstract}. The mentioned articles and writings were used for 
the assignment.

\section{\cite{mejias2011gain} Gain-of-function glutamate receptor interacting protein 1 variants alter GluA2 recycling and surface distribution in patients with autism } Glutamate receptor interacting protein 1 (GRIP1) is a neuronal scaffolding protein that interacts directly with the C termini of glutamate receptors 2/3 (GluA2/3) via its PDZ domains 4 to 6 (PDZ4–6). We found an association (P < 0.05) of a SNP within the PDZ4-6 genomic region with autism by genotyping autistic patients (n = 480) and matched controls (n = 480). Parallel sequencing identified five rare missense variants within or near PDZ4–6 only in the autism cohort, resulting in a higher cumulative mutation load (P = 0.032). Two variants correlated with a more severe deficit in reciprocal social interaction in affected sibling pairs from proband families. These variants were associated with altered interactions with GluA2/3 and faster recycling and increased surface distribution of GluA2 in neurons, suggesting gain-of-function because GRIP1/2 deficiency showed opposite phenotypes. Grip1/2 knockout mice exhibited increased sociability and impaired prepulse inhibition. These results support a role for GRIP in social behavior and implicate GRIP1 variants in modulating autistic phenotype.

\section{\cite{feng2009organization} Organization and dynamics of PDZ-domain-related supramodules in the postsynaptic density}
As the major components of the postsynaptic density of excitatory neuronal synapses, PDZ-domain-containing scaffold proteins regulate the clustering of surface glutamate receptors, organize synaptic signalling complexes, participate in the dynamic trafficking of receptors and ion channels, and coordinate cytoskeletal dynamics. These scaffold proteins often contain multiple PDZ domains, with or without other protein-binding modules, and they usually lack intrinsic enzymatic activities. Recent biochemical and structural studies have shown that tandemly arranged PDZ domains often serve as structural and functional supramodules that could regulate the organization and dynamics of synaptic protein complexes, thus contributing to the broad range of neuronal activity.

\section{\cite{takamiya2008glutamate} The glutamate receptor-interacting protein family of GluR2-binding proteins is required for long-term synaptic depression expression in cerebellar Purkinje cells}
Glutamate receptor-interacting protein 1 (GRIP1) and GRIP2 are closely related proteins that bind GluR2-containing AMPA receptors and couple them to structural and signaling complexes in neurons. Cerebellar long-term synaptic depression (LTD) is a model system of synaptic plasticity that is expressed by persistent internalization of GluR2-containing AMPA receptors. Here, we show that genetic deletion of both GRIP1 and GRIP2 blocks LTD expression in primary cultures of mouse cerebellar neurons but that single deletion of either isoform allows LTD to occur. In GRIP1/2 double knock-out Purkinje cells, LTD can be fully rescued by a plasmid-driving expression of GRIP1 and partially rescued by a GRIP2 plasmid. These results indicate that the GRIP family comprises an essential molecular component for cerebellar LTD.
Disks large homolog 3 (DLG3) also known as neuroendocrine-DLG or synapse-associated protein 102 (SAP-102) is a protein that in humans is encoded by the DLG3 gene.[3][4] DLG3 is a member of the membrane-associated guanylate kinase (MAGUK) superfamily of proteins.
\section{\cite{dong1997grip} GRIP: a synaptic PDZ domain-containing protein that interacts with AMPA receptors}
AMPA glutamate receptors mediate the majority of rapid excitatory synaptic transmission in the central nervous system1,2 and play a role in the synaptic plasticity underlying learning and memory3,4. AMPA receptors are heteromeric complexes of four homologous subunits (GluRl–4) that differentially combine to form a variety of AMPA receptor subtypes1,2. These subunits are thought to have a large extracellular amino-terminal domain, three transmembrane domains and an intracellular carboxy-terminal domain5. AMPA receptors are localized at excitatory synapses and are not found on adjacent inhibitory synapses enriched in GABAA receptors6. The targeting of neurotransmitter receptors, such as AMPA receptors, and ion channels to synapses is essential for efficient transmission7,8. A protein motif called a PDZ domain is important in the targeting of a variety of membrane proteins to cell–cell junctions including synapses8–10. Here we identify a synaptic PDZ domain-containing protein GRIP (glutamate receptor interacting protein) that specifically interacts with the C termini of AMPA receptors. GRIP is a new member of the PDZ domain-containing protein family which has seven PDZ domains and no catalytic domain. GRIP appears to serve as an adapter protein that links AMPA receptors to other proteins and may be critical for the clustering of AMPA receptors at excitatory synapses in the brain.



\section{\cite{kim2004pdz} PDZ domain proteins of synapses}
PDZ domains are protein-interaction domains that are often found in multi-domain scaffolding proteins. PDZ-containing scaffolds assemble specific proteins into large molecular complexes at defined locations in the cell. In the postsynaptic density of neuronal excitatory synapses, PDZ proteins such as PSD-95 organize glutamate receptors and their associated signalling proteins and determine the size and strength of synapses. PDZ scaffolds also function in the dynamic trafficking of synaptic proteins by assembling cargo complexes for transport by molecular motors. As key organizers that control synaptic protein composition and structure, PDZ scaffolds are themselves highly regulated by synthesis and degradation, subcellular distribution and post-translational modification.