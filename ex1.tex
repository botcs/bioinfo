\chapter{Search my protein in databases}
My choice was the \textbf{Discs Large MAGUK Scaffold Protein - DLG3}.
I use Google for researching on-line, mainly found matching content on:
\begin{itemize}
\item \href{Uniprot.org}{http://www.uniprot.org/uniprot/Q62936}
\item \href{GeneCards.org}{http://www.genecards.org/cgi-bin/carddisp.pl?gene=DLG3}.
\end{itemize}

\section{What is its function?}
\begin{description}
\item[Entrez Gene Summary for DLG3 Gene]

    This gene encodes a member of the membrane-associated guanylate kinase protein family. The encoded protein may play a role in clustering of NMDA receptors at excitatory synapses. It may also negatively regulate cell proliferation through interaction with the C-terminal region of the adenomatosis polyposis coli tumor suppressor protein. Mutations in this gene have been associated with X-linked mental retardation. Alternatively spliced transcript variants have been described. [provided by RefSeq, Oct 2009]

\item[GeneCards Summary for DLG3 Gene]

DLG3 (Discs Large MAGUK Scaffold Protein 3) is a Protein Coding gene. Diseases associated with DLG3 include Non-Syndromic X-Linked Intellectual Disability and Partington Syndrome. Among its related pathways are Presynaptic function of Kainate receptors and L1CAM interactions. GO annotations related to this gene include ubiquitin protein ligase binding and protein C-terminus binding. An important paralog of this gene is DLG2.

\item[UniProtKB/Swiss-Prot for DLG3 Gene \url{DLG3_HUMAN,Q92796}]

Required for learning most likely through its role in synaptic plasticity following NMDA receptor signaling.

\item[Wikipedia]
Disks large homolog 3 (DLG3) also known as neuroendocrine-DLG or synapse-associated protein 102 (SAP-102) is a protein that in humans is encoded by the DLG3 gene.[3][4] DLG3 is a member of the membrane-associated guanylate kinase (MAGUK) superfamily of proteins.
\end{description}

\section{How many domains does it have and what kinds?}
According to Uniprot, DLG3 gene has \textbf{five} domains, namely the following:
\begin{description}
\item[PDZ 1] Position: 149 – 235
\item[PDZ 1] 244 – 330
\item[PDZ 1] 404 – 484	
\item[SH3] 519 – 589
\item[Guanylate kinase-like]
\end{description}